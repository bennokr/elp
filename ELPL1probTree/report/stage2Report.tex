\documentclass{article}
\usepackage[margin=1cm,a4paper]{geometry}
\usepackage{tikz}
%\usetikzlibrary{trees}
\usepackage{program}

\title{Elements of Language Processing and Learning\\
Lab assignment report\\
Stage 2: Adding support of Unary Rules in the CKY ALgorithm}
\author{Benno Kruit, 10576223\\Sara Veldhoen, 10545298}

\begin{document}
\maketitle

The CKY algorithm was originally implemented for binary rules only. We extended the code to support unary rules.
\paragraph{Implementation}
We created a general class {\tt Rule} to replace the {\tt BinaryRule} and {\tt UnaryRule} classes. Occurrences of {\tt Rule} can be either binary or unary rules. The method {\tt getChildren()} returns an array of Strings. This enables other functions to iterate over the children in a more general way.

The Parsing algorithm itself is run with the function {\tt getBestParse(sentence)}. The pseudocode of the algorithm is as follows:

\begin{program}
\mbox{Match preterminal rules}

\end{program}


\paragraph{Results}

\end{document}
