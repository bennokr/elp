\documentclass{article}
\usepackage[margin=2cm,a4paper]{geometry}
\usepackage{tikz}
%\usetikzlibrary{trees}
\usepackage{program}

\title{Elements of Language Processing and Learning\\
Lab assignment report\\
Stage 2: Adding support of Unary Rules in the CKY ALgorithm}
\author{Benno Kruit, 10576223\\Sara Veldhoen, 10545298}

\begin{document}
\maketitle

The CKY algorithm was originally implemented for binary rules only. We extended the code to support unary rules. To allow for sequences of unary rule applictions, we used the provided class {\tt UnaryClosure}, extending the {\tt Grammar} class to contain all possivle sequences of unary rule applications.

A sentence is only parsed into a tree with unary rules when this leads to a higher probability of the parse. The application of spurious unary rules is not an issue, because thee would always lead to a lower probability.
 
\section{Implementation}
\paragraph{Grammar}
\begin{itemize}
\item We created a general class {\tt Rule} to replace the {\tt BinaryRule} and {\tt UnaryRule} classes. Occurrences of {\tt Rule} can be either binary or unary rules. The method {\tt getChildren()} returns an array of Strings. This enables other functions to iterate over the children in a more general way.

\item We extended the class {\tt Grammar} with {\tt UnaryClosedGrammar}, which is a grammar that computes and stores the unary closure of all unary rules after instantiation. The methods {\tt getUnaryRulesByParent(String)} and {\tt getUnaryRulesByChild} are guaranteed to return the unary closure of unary rules.

\end{itemize}

\paragraph{Parsing}
\begin{itemize}
\item 
The Parsing algorithm itself is run with the function {\tt getBestParse(List<String>)}. The pseudocode of the algorithm is as follows:
\begin{program}
\mbox{match preterminal productions}

\FOR max=1 : 1 :  n
      \FOR min = max-1 : -1 : 0
              \FOREACH C \in syntacticCategories
\FOREACH R \in binary rules
\FOR mid = min+1 : 1: max-1
	          \mbox{match binary rule, keep the best scoring rule and its midpoint}
\END
\END
              \mbox{store the best rule and midpoint in the chart}
\END
 \FOREACH C \in syntacticCategories
\FOREACH R \in unary closure
 \mbox{match unary rule, keep the best scoring rule}
\END
\IF (bestScoringUnary > bestStoredInChart)
\mbox{replace this position in the chart with the unary rule}
           
\END
\END
\END
\END

\end{program}
\item After performing the CKY parsing, the tree is recovered by traversing the backpointers in the chart. 
We altered the classes {\tt Chart} and {\tt EdgeInfo} such that an {\tt EdgeInfo} object has a certain {\tt type}: preterminal, unary, or binary.
An instance of {\tt EdgeInfo} with the preterminal type only needs to store its score. The unary {\tt EdgeInfo} also stores the rule that generated it, and the {\tt EdgeInfo} with the binary type extends this by also storing the midpoint of that rule. When the {\tt set} function of {\tt Chart} is called with 3, 4, or 5 parameters, the appropriate constructor for {\tt EdgeInfo} is called. The parameters always include the {\tt i} and {\tt j}, the end points of the edge, and {\tt label} its label. The optional arguments are a {\tt Rule} object and a {\tt midpoint}.

We rewrote the methods {\tt traverseBackPointers} and {\tt traverseBackPointersHelper} to build the tree recursively:

\begin{program}
\mbox{recursively traverse backpointers, looking up a label $p$ in {\tt chart} between word $i$ and $j$ of {\tt sentence} }

\IF {\tt chart}[i][j][p].type = {\tt preterminal}
	\mbox{return the tree node $p$ with child node {\tt sentence}$[i]$}
\FI
\IF {\tt chart}[i][j][p].type = {\tt unary}
	c \gets \mbox{traverse backpointers for label }{\tt chart}[i][j][p].child \mbox{ between word $i$ and $j$}
	\mbox{return the tree node $p$ with child node $c$}
\FI
\IF {\tt chart}[i][j][p].type = {\tt binary}
	m \gets {\tt chart}[i][j][p].midpoint
	c_1 \gets \mbox{traverse backpointers for label }{\tt chart}[i][j][p].child[1] \mbox{ between word $i$ and $m$}
    c_2 \gets \mbox{traverse backpointers for label }{\tt chart}[i][j][p].child[2] \mbox{ between word $i$ and $m$}
	\mbox{return the tree node $p$ with child nodes $c_1$ and $c_2$}
\FI

\end{program}

\end{itemize}

\section{Results}

We trained the parser on the standard training set, and tested on 1034 trees. For both training and testing, a maximum length of sentences was used of 20 words. The results can be seen below:


\begin{tabular}{| l | l |}
    \hline
    Precision & 79.56 \\ \hline
    Recall & 70.55 \\ \hline
    F1 & 74.79\\ \hline
    EX & 15.37\\ \hline
\end{tabular}


\end{document}
